\addcontentsline{toc}{chapter}{Symbols and Acronyms}
\chapter*{Symbols and Acronyms}

In general, we denote a vector and a matrix by a lower and a upper case bold letter, respectively, e.g., $\mathbf{v} \in \mathbb{R}^n$ and $\mathbf{M} \in \mathbb{R}^{p \times q}$.
An exception to this notation is the use of the letter ``p''.
We use bold uppercase $\mathbf{P}$ and lowercase $\mathbf{p}$ to represent points in the Cartesian space and their projections on the image plane, respectively.
A quantity following $\Delta$ or having $\sim$ above it represents its difference or error.
And a quantity having $\widehat{\ }$ above it represents its estimation.
A time varying quantity is followed by $(t)$.
Quaternions are denoted by an italic letter with a circle above it, e.g., $\mathring{q} = q_0 + q_1\mathbf{i} + q_2\mathbf{j} + q_3\mathbf{k} = (q_0,~\mathbf{q})$.
Leading superscripts identify which coordinate system a quantity is written in, e.g., $^{A}\mathbf{P}$ represents a position vector described in $\{A\}$.
A quantity also possessing a leading subscript specifies a relationship between two coordinate systems, e.g., $^{A}_{B}\mathbf{R}$ and $^{A}_{B}\mathbf{T}$ are respectively rotation and homogeneous transformation matrices from $\{A\}$ to $\{B\}$ [\cite{craig2005introduction}].
Major symbols and acronyms are defined as follows:

\newpage

\noindent
$E \in \mathbb{R}$ \hfill energy \\
$m \in \mathbb{R}$ \hfill mass \\
$c \in \mathbb{R}$ \hfill the speed of light \\
